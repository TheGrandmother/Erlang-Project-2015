\chapter{Reflektioner}

\label{Reflektioner} 

\lhead{Appendix D. \emph{Reflektioner}}

De flesta i gruppen hade inte någon större erfarenhet av Erlang förutom dem problemen vi fick lösa i problem set 3 i den här kursen. Sedan har vi ju även använt Python för det grafiska gränssnittet, Python var de flesta väldigt obekanta med så även det språket har samtliga i gruppen fått en bättre förståelse för. 

Utöver de språkliga kunskaper vi fått, är något vi kommer ta med oss från det här projektet hur man bäst kan utnyttja dynamiken i gruppen för att få arbete gjort så effektivt som möjligt. Tilldela de som är säkra på hur något moment skall göras med någon som är mindre erfaren och på så sätt förbereda denna för nästa liknande moment i projektet. Detta hade vi kunnat implementerat mer, men för att få detta tillväga gångssätt att fungera så måste alla i gruppen vara öppna och förmedla detta till samtliga i gruppen för att gruppen sedan ska bestämma en fungerande uppdelning av gruppen inför momentet. 

Vi har även varit noga med att utnyttja Scrum-metoden, med många korta möten där alla delat vad de haft för problem under veckan och vad de håller på med i dagsläget. Detta har varit väldigt hjälpsamt vid utvecklingen av systemet.

Ett streck i räkningen har varit att hålla samtliga i gruppen uppdaterad med vad för implementation som stundar. Detta har dock inte lett till några förseningar då det varit enkelt att få tag på alla i gruppen om någon haft frågor angående projektet. 
Sedan har gruppen varit svår att samla flera gånger per vecka på grund av krockande scheman, i projektets inledande stadie var vår tanke att vi skulle ses varje vardag under lunchen för att hålla alla uppdateringar och ha en sorts workshop där det är öppet för frågor inom gruppen.

Hade vi kunnat göra om projektet instämmer alla att vi borde ha nyttjat trello mer. Detta skulle ha underlättat synkronisationen av arbetet. Även om vi använde github flitigt så hade vi kunnat använda det till att se till att vi code-reviewade mer frekvent då det fanns många stavfel i koden kvar långt in i projektets gång. 
Som vi nämnde tidigare hade fler dagliga möten varit väldigt praktiskt då det hade gett mer rutin i arbetet. De flesta hade kunnat arbeta samtidigt vilket också hade underlättat kommunikationen

Slutligen hade vi kunnat ha använda Slack mer, då det är väldigt enkelt att citera och dela kod sinsemellan utan att behöva commita eller adda något till git repositoriet. Det kan också användas som en chatclient och det skulle leda till mindre distraktioner än att använda facebook för detta. 