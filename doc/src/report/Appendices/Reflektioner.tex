\chapter{Reflektioner}

\label{Reflektioner} 

\lhead{Appendix D. \emph{Reflektioner}}

De flesta i gruppen hade inte någon större erfarenhet av Erlang förutom de problemen vi fick lösa i problem set 3 i den här kursen. Vi har även använt Python för det grafiska gränssnittet, de flesta var väldigt obekanta med Python så samtliga i gruppen har fått bättre förståelse även för det språket.

Utöver de språkliga kunskaper vi fått, så är något vi kommer ta med oss från det här projektet hur man bäst kan utnyttja dynamiken i gruppen för att få arbete gjort så effektivt som möjligt. Även att para ihop de som är säkra på hur något moment skall genomföras med någon som är mindre erfaren inom området och på så sätt förbereda denna för nästa liknande moment i projektet. Detta hade vi kunnat använda mer, men för att få detta tillvägagångssätt att fungera så måste alla i gruppen vara öppna med sina kunskaper på förhand och på så sätt underlätta vid uppdelningen av arbetet.

Vi har varit noga med att utnyttja Scrum-metoden, många korta möten där alla berättat vad de haft för problem under veckan och vad de arbetar med i dagsläget. Detta har varit väldigt hjälpsamt vid utvecklingen av systemet.

Ett streck i räkningen har varit att hålla samtliga i gruppen uppdaterad med vad för implementation som stundar. Detta har dock inte lett till några förseningar då det varit enkelt att få tag på alla i gruppen om någon haft frågor angående projektet. 

Gruppen varit svår att samla flera gånger per vecka på grund av schemakrockar. I projektets inledande stadie var vår tanke att vi skulle ses varje vardag under lunchen för att hålla alla uppdaterade och ha en form av workshop där det skulle vara öppet för frågor.

Hade vi kunnat göra om projektet instämmer alla att vi borde ha nyttjat trello mer. Trello skulle ha underlättat synkroniseringen av arbetet. Även om vi använde github flitigt så hade vi kunnat använda det till att se till att vi code-reviewade mer frekvent då många stavfel fanns kvar i koden långt in under projektets gång. 


Som vi nämnde tidigare hade fler dagliga möten varit väldigt praktiskt då det hade gett mer rutin i arbetet. De flesta hade kunnat arbeta samtidigt vilket också hade underlättat kommunikationen.

Slutligen hade vi kunnat använda Slack mer, då det är väldigt enkelt att citera och dela kod sinsemellan istället för att behöva commita eller adda något till git-repositoriet. Slack kunde också ha använts som en chattklient, vilket skulle ha lett till mindre distraktioner än användandet av facebook-chatten.
