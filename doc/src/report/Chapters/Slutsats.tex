\chapter{Slutsatser}

\label{Slutsats} 

\lhead{Kapitel 4. \emph{Slutsatser}}

Vi har lyckats göra en representation av en myrkoloni som samlar mat och kommunicerar med varandra med hjälp av feromoner. Myrorna kan även undvika olika hinder och objekt.  Även om vi inte riktigt lyckades implementera allt vi hade i åtanke när projektet började så är vi nöjda med slutresultatet. Det är svårt att få en verklig uppfattning av problemen och utmaningarna med ett arbete innan man satt igång med grunden på vilken andra implementationer ska vila. 
Vi bestämde oss för att se till att göra en grundläggande representation av en myrkoloni och se till att denna har god concurrency och en felfri körning, detta har vi uppnåt och även utvecklat en grafisk representation  Det är detta som vi nämnde tidigare, som är vår grund. På denna grund kan man enkelt bygga ut och genom detta implementera flera funktioner, som till exempel flera olika typer av mat, fientliga insekter och en konkurrerande myrkoloni.
Även om vår första tanke var att använda det nya och lite mer spännande språket Nim, så är vi nöjda med att vi valde att använda det mer utvecklade och etablerade språket Erlang. De verktyg som Erlang har tillgängliga har varit hjälpsamma och det är ett intressant språk att skriva i och actor-modellen är lätthanterlig för den här typen av concurrency.  

Sammanfattningsvis så har systemet levt upp till de förväntningar som gruppen hade haft i projektets tidiga stadie. 
