\documentclass[a4paper]{article}

\usepackage[english]{babel}
\usepackage[utf8x]{inputenc}
\usepackage{amsmath}
\usepackage{graphicx}
\usepackage[colorinlistoftodos]{todonotes}
\usepackage[margin = 1.2in]{geometry}

\title{A Bug's Life}
\author{Oliver Eriksson Edholm (XXXXXX-YYYY) \\
		Aleksander Lundqvist (XXXXXX-YYYY) \\
		Henrik Sommerland (890618-4950) \\
		Edvin Wahlberg (XXXXXX-YYYY) \\
		Oscar Wallster(910615-1096)}

\begin{document}
\maketitle

\section{Introduction}
%This could be rewritten in a far more persuasive manner
We have decided that we are going to simpulate an \emph{ant ant}. We want to do this in order to learn
more about both \emph{swarm intelligence} and the actor model.\\
We choose ants in inspiration of their ability to cooperate in a very large scale and solve complex problems while still every ant is only following very simple rules. 
\\
Our goal is not to simulate the behaviour of real world ants or to mimick the datails of any biological systems.
We are more interested in the theorethical concepts of how complex behaviors can arise from the interaction of
a large group of agent each possesing only limited cognitive abilites.\\
\\
One of the later goals is also to have two or more ant hives interact with eachother in a world where food is scarce, and battles are inevidieble.

\subsection{Chalenges}
\subsubsection{Concurrency}
Making heavy use of the actor model using thousands of actors will create a lot of complexity and there are many posibilities for
problems to occur. The most obvious danger is the possibilty of deadlocks to occur. With many actors comunicating together one is
almost certain that one will get some form of circular dependencies. So great care needs to be taken to ensure that no deadlocks will occur.

There is also a risk of severe performance degradation in regions with many interacting actors. Although this is something wich is intrinsic to
the actor model and it may be hard for us to control it.
\subsubsection{Rules For Interaction}
The other chalange will be to setup rules for how the ants interact with the world and jhow the world gets updated. The parameters for these rules will need to be
finetuned and it may be hard to find rules which result in complex behaviours. It is also very hard to analytically determine these parameters so the only
feasible alternative is to trough experimentation find rules which yields satisfactory behavious.
\subsubsection{Social Challanges}
In order to get this project moving forward in a pace that is required, we need to have every member of the group working regularly certain hours of the weeks. In order to achieve this every member needs to have the same mental image over the finished project, therefore we need to be active with communication in the group so that noone is left behind and wonders what he could/should be doing. 

\section{Concurrency Models and Programming Languages}
\subsection{Concurrency Goals}
As previously stated we want to create a system wich relies heavily on the actor model. Our project idea
is well suited for this since the ants are by nature individual agents who and make decisions without any direct knowlegde
of the global state.
\subsubsection{No Global Locks}
We want to make our simulation free of so called \emph{stop the world}
sencarios. These are when we for some reason have to block al of the actors in
the simulation in order to perform some kind of action.


\subsection{Rust}

\subsection{Nim}

\subsection{Erlang}

\subsection{Encore}

\section{System Architecture}

\section{Development Tools}
In order to achieve this every member needs to have the same mental image over the finished project, therefore we need to be active with communication in the group so that noone is left behind and wonders what he could/should be doing. 

\section{Concurrency Models and Programming Languages}
\subsection{Concurrency Goals}
As previously stated we want to create a system wich relies heavily on the actor model. Our project idea
is well suited for this since the ants are by nature individual agents who and make decisions without any direct knowlegde
of the global state.
\subsubsection{No Global Locks}
We want to make our simulation free of so called \emph{stop the world}
sencarios. These are when we for some reason have to block al of the actors in
the simulation in order to perform some kind of action.


\subsection{Rust}

\subsection{Nim}

\subsection{Erlang}

\subsection{Encore}

\section{System Architecture}

\section{Development Tools}
Communication and planning: To plan meetings and set up to-do tasks for the project, we have chosen to use Trello. Trello is very user friendly and will give us a great overview of the progress of the project. It will also be a good tool to keep track of who is doing what. For communication we’re going to use Slack to ensure that we have all communication of the project in one place. If we should have chosen to use Facebook it’s very easy to get distracted and talk about other things that isn’t related to the project. If we really need discuss something urgent we’re going to use Skype.
\section{Conclusion}
Humans have alot to learn from ants, 
\end{document}

